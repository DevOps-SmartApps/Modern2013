%% From Locke to Freud: mindmap
\pdfbookmark[2]{From Locke to Freud: Art, Philosophy, Science and History. Mindmap}{Lock2Freud}
\phantomsection
\addcontentsline{toc}{subsection}{From Locke to Freud: Art, Philosophy, Science and History. Mindmap}

%% Adding Mindmap entries to the index
\index{Unification of Germany}
\index{Austro-Prussian War}
\index{Second Schleswig War}
\index{Glorious Revolution}
\index{Haussmann's renovation of Paris}
\index{June Days Uprising}
\index{French Revolution of 1848}
\index{Louis Philippe}
\index{Reign of Terror}
\index{French Revolution}
\index{Congress of Vienna}
\index{Franco-Prussian War}
\index{World War I}
\index{On the Origin of Species}
\index{The Descent of Man}
\index{The Interpretation of Dreams}
\index{Civilization and Its Discontents}
\index{On the Genealogy of Morality}
\index{The Communist Manifesto}
\index{What is Enlightenment?}
\index{Discourse on Inequality}
\index{Discourse on the Arts and Sciences}
\index{Le Spleen de Paris}
\index{Madame Bovary}
\index{Impressionism}
\index{Realism}
\index{Romantiscism}
\index{Liberty Leading the People}
\index{The Massacre at Chios}
\index{Neoclassicism}

\hspace*{1cm}\copyrightboxOddEven{%
\begin{tikzpicture}[scale=0.95,mindmap,
every node/.style={concept,execute at begin node=\hskip0pt},
root concept/.append style={
concept color=black, fill=white, line width=1ex, text=black,minimum size=3cm,text width=3cm,font=\large\scshape},
text=historyfg,
concept color=historybg,grow cyclic,
level 1/.append style={level distance=3.7cm,sibling angle=90},
level 2/.append style={level distance=3cm,sibling angle=45},
level 3/.append style={level distance=2.3cm,sibling angle=45}]
\node[root concept] {Modernity: From Locke to Freud}
  child [concept,sibling angle=95,color=historybg,text=historyfg] { node {History}
	child[sibling angle=50] { node {Germany}
		child { node {\hrefmm{http://en.wikipedia.org/wiki/Unification_of_Germany}{historyfg}{1871: Unification of Germany}}}
		child { node {\hrefmm{http://en.wikipedia.org/wiki/Austro-Prussian_War}{historyfg}{1866: Austro-Prussian War}}}
		child { node {\hrefmm{http://en.wikipedia.org/wiki/Second_Schleswig_War}{historyfg}{1864: Second Schleswig War}}}
	}
	child[sibling angle=67] { node {England} 
		child { node {\hrefmm{http://en.wikipedia.org/wiki/Glorious_Revolution}{historyfg}{1688 -- 1689: Glorious Revolution}}}
	}
	child[sibling angle=72,level distance=3.8cm] { node {France} 
		child { node {\hrefmm{http://en.wikipedia.org/wiki/Haussmann\%27s_renovation_of_Paris}{historyfg}{1853--1870: Hauss\-mann's renovation of Paris}}}
		child { node {\hrefmm{http://en.wikipedia.org/wiki/June_Days_Uprising}{historyfg}{1848: June Days Uprising}}}
		child { node {\hrefmm{http://en.wikipedia.org/wiki/Revolutions_of_1848_in_France}{historyfg}{1848: French Revolution of 1848}}}
		child { node {\hrefmm{http://en.wikipedia.org/wiki/Louis_Philippe_I}{historyfg}{1830--1848: Louis Philippe}}}
		child { node {\hrefmm{http://en.wikipedia.org/wiki/Reign_of_Terror}{historyfg}{1793--1794: Reign of Terror}}}
		child { node {\hrefmm{http://en.wikipedia.org/wiki/French_revolution}{historyfg}{1789: French Revolu\-tion}}}
	}
	child[sibling angle=70] { node {Europe} 
		child { node {\hrefmm{http://en.wikipedia.org/wiki/Vienna_congress}{historyfg}{1814--1815: Congress of Vienna}}}
		child { node {\hrefmm{http://en.wikipedia.org/wiki/Franco-Prussian_War}{historyfg}{1870--1871: Franco-Prussian War}}}
		child { node {\hrefmm{http://en.wikipedia.org/wiki/World_War_I}{historyfg}{1914-1918: World War I}}}
	}	
  }
  child [concept color=scibg,text=black,sibling angle=70] { node {Science}
	child { node {\hrefmm{http://en.wikipedia.org/wiki/On_the_Origin_of_Species}{black}{1859: On the Origin of Species}}}
	child { node {\hrefmm{http://en.wikipedia.org/wiki/Descent_of_man}{black}{1871: The Descent of Man}}}
	child { node {\hrefmm{http://en.wikipedia.org/wiki/The_Interpretation_of_Dreams}{black}{1899: The Interpre\-tation of Dreams}}}
  }
  child [concept color=historybg,text=historyfg,level distance=5cm] { node { Philosophy}
  	child { node {\hrefmm{http://en.wikipedia.org/wiki/Civilization_and_Its_Discontents}{historyfg}{1930: Civilization and Its Discontents}}}
	child { node {\hrefmm{http://en.wikipedia.org/wiki/On_the_Genealogy_of_Morals}{historyfg}{1887: On the Genealogy of Morality} }}
	child { node {\hrefmm{http://en.wikipedia.org/wiki/The_Communist_Manifesto}{historyfg}{1848: The Communist Manifesto} }}
	child { node {\hrefmm{http://en.wikipedia.org/wiki/Answering_the_Question:_What_is_Enlightenment\%3F}{historyfg}{1784: What is Enlighten\-ment?}}}
	child { node {\hrefmm{http://en.wikipedia.org/wiki/Discourse_on_the_Origin_of_Inequality}{historyfg}{1755: Discourse on Inequality}}}
	child[sibling angle=48] { node {\hrefmm{http://en.wikipedia.org/wiki/Discourse_on_the_Arts_and_Sciences}{historyfg}{1750: Discourse on the Arts and Sciences}}}
  }
  child [concept color=scibg,text=black] { node {Arts}
  	child [concept, sibling angle=70 ] { node {Literature}
  		child { node {\hrefmm{http://en.wikipedia.org/wiki/Paris_Spleen}{black}{1869: Le Spleen de Paris}}}
		child { node {\hrefmm{http://en.wikipedia.org/wiki/Madame_Bovary}{black}{1856: Madame Bovary}}}
  	}
	child [concept ] { node {Paintings}
		child[sibling angle=40] { node {\hrefmm{http://en.wikipedia.org/wiki/Impressionism}{black}{Impres\-sionism}}
  			child { node {\hrefmm{http://en.wikipedia.org/wiki/Olympia_(Manet)}{black}{1863: Olympia } }}
		}
		child { node {\hrefmm{http://en.wikipedia.org/wiki/Realism_(visual_arts)}{black}{Realism}}
			child { node {\hrefmm{http://en.wikipedia.org/wiki/The_Stone_Breakers}{black}{1850: The Stone Breakers}}}
		}
		child[sibling angle=35] { node {\hrefmm{http://en.wikipedia.org/wiki/Romanticism}{black}{Romantis\-cism}}
			child [sibling angle=55] { node {\hrefmm{http://en.wikipedia.org/wiki/Liberty_Leading_the_People}{black}{1830: Liberty Leading the People } }}
			child[sibling angle=60] { node {\hrefmm{http://en.wikipedia.org/wiki/The_Massacre_at_Chios}{black}{1824: The Massacre at Chios} }}
		}
		child[sibling angle=37] { node {\hrefmm{http://en.wikipedia.org/wiki/Neoclassicism}{black}{Neoclas\-sicism}} }
  	}
  };
\end{tikzpicture}}


