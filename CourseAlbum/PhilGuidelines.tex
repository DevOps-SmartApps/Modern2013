\subsection*{Howard Levine. Guidelines for philosophical writing}
\addcontentsline{toc}{subsection}{Howard Levine. Guidelines for philosophical writing}

\begin{flushright}
\parbox{7cm}{\sl Writing is easy; all you do is sit staring at a blank sheet of paper until drops of blood form on your forehead.}

--- Gene Fowler

\bigskip

\parbox{7cm}{\sl Everybody likes being an author; it’s the writing they can’t stand.} 

--- HPL
\end{flushright}

\begin{multicols}{2}
\begin{enumerate}
\item  Create an imaginative and accurate title for your paper. This title should help orient the reader and focus your thoughts. Untitled may be the most popular title for paintings (a philosophical paradox) but it doesn’t work as an organizing idea for a paper. Compare your reaction to the following three titles: Symmetry; Shapes, Space, and Symmetry; and Symmetrical Canvases: How Is It Possible For a Museum to Hang a Painting Upside Down?
\item  Announce in your opening paragraph the thesis of the paper. For example, compare the impacts of the following two opening sentences: “Roy DeForest is an American artist who lives in the Bay Area” vs. “Roy DeForest’s use of cartoon imagery breaks down the distinction between ‘high’ and ‘low’ art.” Explicitly state your aims and the argumentative strategy you will use to support them. For example, your strategy might be to argue horizontally by pressing one key example or case and claiming that what is true of your example holds across the board. Alternatively, you might argue vertically working from a general principle down to specific cases. Of course, you can also employ both. Be sure to identify clearly which arguments are yours. Take a stand on the issues early in the paper and continue to express your ideas. Remember, it is not enough just to state your beliefs. You must argue for them by presenting evidence. Your job is to convince the reader that your beliefs (i.e., that Jeff Koon is overrated) should be his.
\item  Use topic sentences (e.g., Jackson Pollock developed his drip painting technique while painting murals with Orozco in the 1930s ) and then amplify the thought throughout the paragraph. Employ “segue” words to show the relationships between sentences (e.g., ‘That is’, ‘For example’, ‘However’, ‘Therefore’, ‘On the contrary’, ‘Finally’, ‘In conclusion’. Similarly, consider the segue between paragraphs. Ultimately, a good paper starts with one good sentence, links it to the next, and continues to build until the argument is successfully made. In order to accomplish this goal, you need to pay attention to both the sentences and the logic that unites them.
\item    Be specific. Use concrete examples to illustrate and illuminate abstract, general points. One major difference between writing that comes alive, and boring, pedantic writing is that the former shows by example while the latter simply tells. For example, compare: “The cylindrical neck volumes in Roy Carruthers’s paintings, reminiscent of a pirate’s peg leg, contrast sharply with his perfectly proportioned ordinary household objects” vs. “Roy Carruthers’ uses proportion differently for people and objects.” Avoid using metaphors (i.e., inexplicit comparisons such as “Picasso played the bull to Braque’s matador”) unless you accompany them with another sentence that makes explicit the point you wish to make by introducing the metaphor. Don’t rely on your reader’s interpretation of your metaphor.
\item    Use extreme words (e.g., ‘all’, ‘always’, ‘never’, ‘none’, ‘every’, ‘must’, ‘absolutely’, ‘unquestionable’) with extreme caution. Proving a universal (e.g., “All computer art is derivative’) is extremely difficult; proving a negative (e.g., “No one can create a sculpture in four dimensions”) is equally daunting. Avoid hyperbole (e.g., “Duchamp is unquestionably the greatest artist since Michelangelo”).
\item  Define all the technical and special terms you use. All fields, and art is no exception, have a specialized vocabulary that may need elucidation (e.g., everything from technical terms like ‘gouache’ and ‘chiaroscuro’ to movements such as constructivism and conceptual art). Do not assume that your reader has your background or that she understands the term in the exact same way that you do. Also beware of common, everyday terms that may have a special meaning in the art world (e.g.,’mannerism’ vs. ‘Mannerism’ or ‘naive art’.)
\item  Anticipate counter arguments to your view and pre-empt them by modifying your position or by demonstrating why the counter argument is weak. For example, suppose you are arguing that Christo is insignificant because it is scale and not aesthetic value that calls him to our attention. You will need to deal with the counter argument that scale is an integral part of aesthetic value rather than something independent of it. Don’t construct counter arguments that are merely straw men set up so that you can readily knock them down (e.g., Suppose you wish to argue that abstraction is the major theme of twentieth century art. It would be foolish to argue against this thesis simply because abstraction did not begin at the exact dawn of the twentieth century.) Beware of sketchy arguments (e.g., Koon is a hack exactly because of his commercial success). If an argument is worth making, it is worth making in sufficient detail (i.e., what is the relationship between artistic merit and commercial success? What reason is there to think that it is an inverse relationship?). Treat your opponents with the principle of charity -- try revising opposing views to deal with your objections. The more interesting and successful the argument, the more space that should be devoted to it.
\item    Minimize assumptions, especially key, controversial, or unstated assumptions. For example, writing that ”Picasso’s greatness as an artist is undermined because of his misogyny” as if it were an unalloyed truth not in need to argument or support would only demonstrate to the reader your prejudices. Be especially careful not to make unsupportable claims such “everyone knows that...” or “as most everyone believes...” You need to explicitly argue for most everything you say.
\item  Use a footnote (precisely citing the source) to credit others when you use their views or arguments. You need not credit anyone teaching this course, however. (On the other hand, you might want to blame them). Avoid quote-quilting (i.e., overusing others’ arguments and merely weaving them into a position). Give some original arguments, put some arguments into your own words, and refer to, and employ, helpful arguments put forward by others. Finally, make certain that your bibliography conforms to proper rules.
\item  Carefully proofread your paper. At best, typographical or grammatical errors distract your reader and dividing your reader’s attention risks misinterpretation. At worst, such errors obscure the thoughts you wish to convey, may confuse your reader, and tend to convince him that his wisdom is no match for your ignorance. Use a spelling checker but remember it will not distinguish between ‘there’, ‘their’, and ‘they’re’. There is no substitute for careful human scrutiny of your work; better that scrutiny should come from you (or a friend) before I turn my editing pencil loose.
\end{enumerate}
\end{multicols}
\newpage
