\documentclass{article}
\usepackage[T1,T2A]{fontenc}
\usepackage[utf8]{inputenc}
\usepackage[french,english]{babel}
\usepackage[pdftex]{graphicx}
\usepackage{cmap}
\usepackage{hyperxmp}
\usepackage[landscape,left=0.1cm, right=.1cm, top=0.1cm, bottom=.1cm,paperwidth=8in,paperheight=18in]{geometry}
\usepackage[hidelinks]{hyperref}
\usepackage[usenames,dvipsnames]{xcolor}
\usepackage{ccicons}
\usepackage{tikz}
\usepackage{calc}
\usetikzlibrary{calc}
\usetikzlibrary{shapes.callouts}

\begin{document}

%% PDF meta-data
\hypersetup{%
pdftitle={Timeline: Modern and Postmodern key events.},
pdfauthor={Vitaly Repin},
pdfcopyright={This work is licensed under a Creative Commons Attribution-ShareAlike 3.0 Unported License},
pdfsubject={The Modern and the Postmodern},
pdfkeywords={modern,postmodern,tikz,history,timeline},
pdflicenseurl={http://creativecommons.org/licenses/by-sa/3.0/},
pdfcaptionwriter={Vitaly Repin},
pdfcontactcity={Espoo},
pdfcontactcountry={Finland},
pdfcontactemail={vitaly.repin@gmail.com},
}

%% To put hyperlinks into the nodes (whole area, not only texts)
%% Taken from: http://tex.stackexchange.com/questions/36109/making-tikz-nodes-hyperlinkable
\tikzset{
   hyperlink node/.style={
       alias=sourcenode,
       append after command={
           let     \p1 = (sourcenode.north west),
               \p2=(sourcenode.south east),
               \n1={\x2-\x1},
               \n2={\y1-\y2} in
           node [inner sep=0pt, outer sep=0pt,anchor=north west,at=(\p1)] {\href{#1}{\phantom{\rule{\n1}{\n2}}}}
       }
   }
}

%% Puts event to the timeline
%% Arguments:
%% Arg1: Start Year
%% Arg2: End Year
%% Arg3: Description
%% Arg4: Color
%% Arg5: URL
%% Arg6: Text widdth
%% Optional argument: UP (1) or DOWN (-1)
\newcommand{\putEvent}[7][1]{
%% Starting position
\pgfmathsetlengthmacro{\startPos}{(((#2>\constStartYear) ? #2 : \constStartYear) - \constStartYear)*\constMminYear}
%% End position
\pgfmathsetlengthmacro{\endPos}{(((#3<\constStopYear) ? #3 : \constStopYear) - \constStartYear)*\constMminYear}
%% Center position
\pgfmathsetlengthmacro{\centerPos}{(\endPos + \startPos)/2}
\pgfmathsetlengthmacro{\yPos}{#1*(\constLifeSpanHeight+0.8cm)}
%% Drawing circle if event is 1-year event. Otherwise, draw a bar
\ifnum\pdfstrcmp{#2}{#3}=0
\node[fill=#5,circle,inner sep=0pt, anchor=center,minimum width=\constMminYear / 2, minimum height=\constLifeSpanHeight / 2] at (\centerPos, 0) {};
\def\pictTitle{#2: #4}
\else
%% Width of the event
\pgfmathsetlengthmacro{\spanWidth}{\endPos-\startPos}
\node[rectangle,rounded corners,fill=#5,inner sep=0pt,minimum height=\constLifeSpanHeight/2, minimum width=\spanWidth, anchor=south west,hyperlink node={#6}] at (\startPos, 0) {};
\def\pictTitle{#2 -- #3: #4}
\fi
\ifnum\pdfstrcmp{#7}{-1}=0
\pgfmathsetlengthmacro{\mytxtwidth}{2mm + \widthof{\pictTitle}}
\else
\pgfmathsetlengthmacro{\mytxtwidth}{#7}
\fi
\pgfmathsetlengthmacro{\xpos}{\xshift+\centerPos}
\node[rectangle callout, rounded corners, text width=\mytxtwidth, callout absolute pointer={(\centerPos,#1*\constLifeSpanHeight / 2 )},anchor=center, draw=#5] at (\xpos, \yPos) {\textcolor{black}{\href{#6}{\pictTitle}}};
\pgfmathsetlengthmacro{\xshift}{0cm}
}

%% Draws vertical tics for the timeline
%% Arg1: Year
\newcommand{\drawVertTics}[1]{%
%% X pos
\pgfmathsetlengthmacro{\pos}{(#1 - \constStartYear)*\constMminYear}
\draw[ultra thin,densely dotted] (\pos, 1cm) -- (\pos, .9\textheight);
\node at (\pos, 0) {#1};
}

%% Painters
\colorlet{lifespancolorpa}{ForestGreen}
%% Event color
\colorlet{eventcolor}{Brown}
%% Paper color
\colorlet{papercolor}{MidnightBlue}

\newcommand{\colsign}[1]{\colorbox{#1}{\hspace*{1cm}}}
\newsavebox{\legend}
\savebox{\legend}{\Large\itshape\parbox{7cm}{%
\begin{itemize}
\item[\colsign{eventcolor}] Historical events
\item[\colsign{papercolor}] Papers
\item[\colsign{lifespancolorpa}] Paintings
\end{itemize}}}

%% Updates shift
\newcommand{\updateshift}[1][3.3cm]{\pgfmathsetlengthmacro{\myshift}{\myshift + #1}}

\noindent\begin{tikzpicture}[x=1mm,y=1mm]
\pgfmathsetlengthmacro{\xshift}{0mm}
\pgfmathsetlengthmacro{\constLifeSpanHeight}{3mm}
\pgfmathsetmacro{\constStartYear}{1740}%
\pgfmathsetmacro{\constStopYear}{1984}%
\pgfmathsetlengthmacro{\myshift}{2.5cm}%
%% Scale: points in one year
\pgfmathsetlengthmacro{\constMminYear}{.9\textwidth/(\constStopYear-\constStartYear)}
%% Grid
\foreach \i in {1740,1760,...,1980} {
\drawVertTics{\i}
}
\begin{scope}[yshift=\myshift]
\pgfmathsetlengthmacro{\xshift}{-.7cm}
\putEvent{1789}{1789}{French Revolution}{eventcolor}{http://en.wikipedia.org/wiki/French_revolution}{-1}
\pgfmathsetlengthmacro{\xshift}{-1.5cm}
\putEvent[-1]{1793}{1794}{Reign of Terror}{eventcolor}{http://en.wikipedia.org/wiki/Reign_of_Terror}{-1}
\pgfmathsetlengthmacro{\xshift}{.3cm}
\putEvent{1814}{1815}{Congress of Vienna}{eventcolor}{http://en.wikipedia.org/wiki/Vienna_congress}{-1}
\pgfmathsetlengthmacro{\xshift}{-2cm}
\putEvent[-1]{1830}{1848}{July Monarhcy in France (Louis Philippe)}{eventcolor}{http://en.wikipedia.org/wiki/Louis_Philippe_I}{-1}
\putEvent{1864}{1864}{Second Schleswig War}{eventcolor}{http://en.wikipedia.org/wiki/Second_Schleswig_War}{-1}
\putEvent[-1]{1871}{1871}{Unification of Germany}{eventcolor}{http://en.wikipedia.org/wiki/Unification_of_Germany}{-1}
\pgfmathsetlengthmacro{\xshift}{-.7cm}
\putEvent{1914}{1918}{World War I}{eventcolor}{http://en.wikipedia.org/wiki/World_War_I}{-1}
\putEvent[-1]{1918}{1920}{1918 flu pandemic}{eventcolor}{http://en.wikipedia.org/wiki/1918_flu_pandemic}{-1}
\putEvent{1939}{1945}{World War II}{eventcolor}{http://en.wikipedia.org/wiki/World_War_II}{-1}
\pgfmathsetlengthmacro{\xshift}{1.5cm}
\putEvent[-1]{1947}{1947}{Dialectic of Enlightenment}{papercolor}{http://en.wikipedia.org/wiki/Dialectic_of_Enlightenment}{-1}
\end{scope}
\updateshift
\begin{scope}[yshift=\myshift]
\putEvent[-1]{1784}{1784}{Answering the Question: What is Enlightenment?}{papercolor}{http://en.wikipedia.org/wiki/Answering_the_Question:_What_is_Enlightenment\%253F}{-1}
\putEvent[-1]{1853}{1870}{Haussmann's renovation of Paris}{eventcolor}{http://en.wikipedia.org/wiki/Haussmann's_renovation_of_Paris}{-1}
\putEvent[-1]{1921}{1921}{Tractatus Logico-Philosophicus}{papercolor}{https://en.wikipedia.org/wiki/Tractatus_Logico-Philosophicus}{-1}
\putEvent{1927}{1927}{To the Lighthouse}{papercolor}{http://en.wikipedia.org/wiki/To_the_Lighthouse}{-1}
\putEvent[-1]{1969}{1969}{On Certainty}{papercolor}{http://en.wikipedia.org/wiki/On_Certainty}{-1}
\end{scope}
\updateshift[2.2cm]
\begin{scope}[yshift=\myshift]
\putEvent[-1]{1848}{1848}{French Revolution}{eventcolor}{http://en.wikipedia.org/wiki/Revolutions_of_1848_in_France}{-1}
\putEvent{1848}{1848}{June Days Uprising}{eventcolor}{http://en.wikipedia.org/wiki/June_Days_Uprising}{2.7cm}
\putEvent{1866}{1866}{Austro-Prussian War}{eventcolor}{http://en.wikipedia.org/wiki/Austro-Prussian_War}{2.5cm}
\pgfmathsetlengthmacro{\xshift}{1.7cm}
\putEvent[-1]{1870}{1871}{Franco-Prussian War}{eventcolor}{http://en.wikipedia.org/wiki/Franco-Prussian_War}{-1}
\end{scope}
\begin{scope}[yshift=\myshift]
\pgfmathsetlengthmacro{\xshift}{1.8cm}
\putEvent{1750}{1750}{Discourse on the Arts and Sciences}{papercolor}{http://en.wikipedia.org/wiki/Discourse_on_the_Arts_and_Sciences}{-1}
\pgfmathsetlengthmacro{\xshift}{3cm}
\putEvent[-1]{1755}{1755}{Discourse on the Origin and Basis of Inequality Among Men}{papercolor}{http://en.wikipedia.org/wiki/Discourse_on_the_Origin_of_Inequality}{-1}
\putEvent{1824}{1824}{The Massacre at Chios}{lifespancolorpa}{http://en.wikipedia.org/wiki/The_Massacre_at_Chios}{3.8cm}
\putEvent[-2]{1830}{1830}{Liberty Leading the People}{lifespancolorpa}{http://en.wikipedia.org/wiki/Liberty_Leading_the_People}{-1}
\putEvent{1899}{1899}{The Interpretation of Dreams}{papercolor}{http://en.wikipedia.org/wiki/The_Interpretation_of_Dreams}{-1}
\putEvent[2]{1930}{1930}{Civilization and Its Discontents}{papercolor}{http://en.wikipedia.org/wiki/Civilization_and_Its_Discontents}{-1}
\putEvent{1953}{1953}{Philosophical Investigations}{papercolor}{http://en.wikipedia.org/wiki/Philosophical_Investigations}{-1}
\putEvent[-1]{1964}{1964}{Madness and Civilization}{papercolor}{http://en.wikipedia.org/wiki/Madness_and_Civilization}{-1}
\pgfmathsetlengthmacro{\xshift}{-3cm}
\putEvent[2]{1984}{1984}{What is Enlightenment?}{papercolor}{http://en.wikipedia.org/wiki/Answering_the_Question:_What_is_Enlightenment\%253F}{-1}
\end{scope}
\updateshift[4.5cm]
\begin{scope}[yshift=\myshift]
\putEvent[-1]{1887}{1887}{On the Genealogy of Morality}{papercolor}{http://en.wikipedia.org/wiki/On_the_Genealogy_of_Morals}{-1}
\putEvent{1830}{1830}{Self-Reliance}{papercolor}{https://en.wikipedia.org/wiki/Self-Reliance}{-1}
\pgfmathsetlengthmacro{\xshift}{-.8cm}
\putEvent[-1]{1844}{1844}{Experience}{papercolor}{https://en.wikipedia.org/wiki/Experience_(Emerson)}{-1}
\putEvent[2]{1848}{1848}{The Communist Manifesto}{papercolor}{http://en.wikipedia.org/wiki/The_Communist_Manifesto}{-1}
\pgfmathsetlengthmacro{\xshift}{-1.7cm}
\putEvent[-2]{1856}{1856}{Madame Bovary}{papercolor}{http://en.wikipedia.org/wiki/Madame_Bovary}{-1}
\pgfmathsetlengthmacro{\xshift}{2.6cm}
\putEvent[-2]{1859}{1859}{On the Origin of Species}{papercolor}{http://en.wikipedia.org/wiki/On_the_Origin_of_Species}{-1}
\pgfmathsetlengthmacro{\xshift}{-1.6cm}
\putEvent{1869}{1869}{Le Spleen de Paris}{papercolor}{http://en.wikipedia.org/wiki/Paris_Spleen}{3cm}
\pgfmathsetlengthmacro{\xshift}{2.2cm}
\putEvent{1871}{1871}{The Descent of Man}{papercolor}{http://en.wikipedia.org/wiki/Descent_of_man}{-1}
\end{scope}
\node[font=\Huge\scshape\bfseries,fill=blue!20,rounded corners] at (.5\textwidth, 167) {1750 -- 1984};
\node at (55,140) { \usebox{\legend} };
\end{tikzpicture}

\bigskip

\centerline{\ccbysa\ \href{http://www.vrepin.org}{Vitaly Repin}, 2013. Own work.}

\end{document}
